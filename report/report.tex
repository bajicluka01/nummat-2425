\documentclass[12pt,a4paper]{article}

\usepackage[a4paper,text={16.5cm,25.2cm},centering]{geometry}
\usepackage{lmodern}
\usepackage{amssymb,amsmath}
\usepackage{bm}
\usepackage{graphicx}
\usepackage{microtype}
\usepackage{hyperref}
\setlength{\parindent}{0pt}
\setlength{\parskip}{1.2ex}

\hypersetup
       {   pdfauthor = {  },
           pdftitle={  },
           colorlinks=TRUE,
           linkcolor=black,
           citecolor=blue,
           urlcolor=blue
       }




\usepackage{upquote}
\usepackage{listings}
\usepackage{xcolor}
\lstset{
    basicstyle=\ttfamily\footnotesize,
    upquote=true,
    breaklines=true,
    breakindent=0pt,
    keepspaces=true,
    showspaces=false,
    columns=fullflexible,
    showtabs=false,
    showstringspaces=false,
    escapeinside={(*@}{@*)},
    extendedchars=true,
}
\newcommand{\HLJLt}[1]{#1}
\newcommand{\HLJLw}[1]{#1}
\newcommand{\HLJLe}[1]{#1}
\newcommand{\HLJLeB}[1]{#1}
\newcommand{\HLJLo}[1]{#1}
\newcommand{\HLJLk}[1]{\textcolor[RGB]{148,91,176}{\textbf{#1}}}
\newcommand{\HLJLkc}[1]{\textcolor[RGB]{59,151,46}{\textit{#1}}}
\newcommand{\HLJLkd}[1]{\textcolor[RGB]{214,102,97}{\textit{#1}}}
\newcommand{\HLJLkn}[1]{\textcolor[RGB]{148,91,176}{\textbf{#1}}}
\newcommand{\HLJLkp}[1]{\textcolor[RGB]{148,91,176}{\textbf{#1}}}
\newcommand{\HLJLkr}[1]{\textcolor[RGB]{148,91,176}{\textbf{#1}}}
\newcommand{\HLJLkt}[1]{\textcolor[RGB]{148,91,176}{\textbf{#1}}}
\newcommand{\HLJLn}[1]{#1}
\newcommand{\HLJLna}[1]{#1}
\newcommand{\HLJLnb}[1]{#1}
\newcommand{\HLJLnbp}[1]{#1}
\newcommand{\HLJLnc}[1]{#1}
\newcommand{\HLJLncB}[1]{#1}
\newcommand{\HLJLnd}[1]{\textcolor[RGB]{214,102,97}{#1}}
\newcommand{\HLJLne}[1]{#1}
\newcommand{\HLJLneB}[1]{#1}
\newcommand{\HLJLnf}[1]{\textcolor[RGB]{66,102,213}{#1}}
\newcommand{\HLJLnfm}[1]{\textcolor[RGB]{66,102,213}{#1}}
\newcommand{\HLJLnp}[1]{#1}
\newcommand{\HLJLnl}[1]{#1}
\newcommand{\HLJLnn}[1]{#1}
\newcommand{\HLJLno}[1]{#1}
\newcommand{\HLJLnt}[1]{#1}
\newcommand{\HLJLnv}[1]{#1}
\newcommand{\HLJLnvc}[1]{#1}
\newcommand{\HLJLnvg}[1]{#1}
\newcommand{\HLJLnvi}[1]{#1}
\newcommand{\HLJLnvm}[1]{#1}
\newcommand{\HLJLl}[1]{#1}
\newcommand{\HLJLld}[1]{\textcolor[RGB]{148,91,176}{\textit{#1}}}
\newcommand{\HLJLs}[1]{\textcolor[RGB]{201,61,57}{#1}}
\newcommand{\HLJLsa}[1]{\textcolor[RGB]{201,61,57}{#1}}
\newcommand{\HLJLsb}[1]{\textcolor[RGB]{201,61,57}{#1}}
\newcommand{\HLJLsc}[1]{\textcolor[RGB]{201,61,57}{#1}}
\newcommand{\HLJLsd}[1]{\textcolor[RGB]{201,61,57}{#1}}
\newcommand{\HLJLsdB}[1]{\textcolor[RGB]{201,61,57}{#1}}
\newcommand{\HLJLsdC}[1]{\textcolor[RGB]{201,61,57}{#1}}
\newcommand{\HLJLse}[1]{\textcolor[RGB]{59,151,46}{#1}}
\newcommand{\HLJLsh}[1]{\textcolor[RGB]{201,61,57}{#1}}
\newcommand{\HLJLsi}[1]{#1}
\newcommand{\HLJLso}[1]{\textcolor[RGB]{201,61,57}{#1}}
\newcommand{\HLJLsr}[1]{\textcolor[RGB]{201,61,57}{#1}}
\newcommand{\HLJLss}[1]{\textcolor[RGB]{201,61,57}{#1}}
\newcommand{\HLJLssB}[1]{\textcolor[RGB]{201,61,57}{#1}}
\newcommand{\HLJLnB}[1]{\textcolor[RGB]{59,151,46}{#1}}
\newcommand{\HLJLnbB}[1]{\textcolor[RGB]{59,151,46}{#1}}
\newcommand{\HLJLnfB}[1]{\textcolor[RGB]{59,151,46}{#1}}
\newcommand{\HLJLnh}[1]{\textcolor[RGB]{59,151,46}{#1}}
\newcommand{\HLJLni}[1]{\textcolor[RGB]{59,151,46}{#1}}
\newcommand{\HLJLnil}[1]{\textcolor[RGB]{59,151,46}{#1}}
\newcommand{\HLJLnoB}[1]{\textcolor[RGB]{59,151,46}{#1}}
\newcommand{\HLJLoB}[1]{\textcolor[RGB]{102,102,102}{\textbf{#1}}}
\newcommand{\HLJLow}[1]{\textcolor[RGB]{102,102,102}{\textbf{#1}}}
\newcommand{\HLJLp}[1]{#1}
\newcommand{\HLJLc}[1]{\textcolor[RGB]{153,153,119}{\textit{#1}}}
\newcommand{\HLJLch}[1]{\textcolor[RGB]{153,153,119}{\textit{#1}}}
\newcommand{\HLJLcm}[1]{\textcolor[RGB]{153,153,119}{\textit{#1}}}
\newcommand{\HLJLcp}[1]{\textcolor[RGB]{153,153,119}{\textit{#1}}}
\newcommand{\HLJLcpB}[1]{\textcolor[RGB]{153,153,119}{\textit{#1}}}
\newcommand{\HLJLcs}[1]{\textcolor[RGB]{153,153,119}{\textit{#1}}}
\newcommand{\HLJLcsB}[1]{\textcolor[RGB]{153,153,119}{\textit{#1}}}
\newcommand{\HLJLg}[1]{#1}
\newcommand{\HLJLgd}[1]{#1}
\newcommand{\HLJLge}[1]{#1}
\newcommand{\HLJLgeB}[1]{#1}
\newcommand{\HLJLgh}[1]{#1}
\newcommand{\HLJLgi}[1]{#1}
\newcommand{\HLJLgo}[1]{#1}
\newcommand{\HLJLgp}[1]{#1}
\newcommand{\HLJLgs}[1]{#1}
\newcommand{\HLJLgsB}[1]{#1}
\newcommand{\HLJLgt}[1]{#1}


\begin{document}



\section{Metoda konjugiranih gradientov s predpogojevanjem}
Avtor: Luka Bajić


\subsection{Opis problema}
Metoda konjugiranih gradientov je postopek za reševanje linearnega sistema enačb $Ax = b$, ob predpostavki, da je matrika A pozitivno definitna. V našem primeru se ukvarjamo z razpršenimi matrikami, torej matrikami, ki imajo večino elementov ničelnih. Da izboljšamo hitrost konvergence metode konjugiranih gradientov, uporabimo predpogojevanje z matriko $M=LL^T$, kjer spodnjetrikotno matriko $L$ dobimo z nepopolnim razcepom Choleskega. 


Ker imamo opravka z razpršenimi matrikami, je s stališča prostorske zahtevnosti smiselno, da jih hranimo v posebni strukturi, ki hrani samo neničelne elemente in sestoji iz treh vektorjev: vektor  vrednosti v matriki, vektor indeksov po vrsticah in vektor indeksov po stolpcih. Na primer za sledečo matriko:


\[
A=\begin{bmatrix}7 & 1.1 & 0 & 0 & 0 \\ 1.1 & 2 & 0 & 0 & 0 \\ 0 & 0 & 3 & 0 & 3 \\ 0 & 0 & 0 & 0.5 & 0 \\ 0 & 0 & 3 & 0 & 4.2 \end{bmatrix}
\]

hranimo vektorje


\[
I = \begin{bmatrix}1 & 1 & 2 & 2 & 3 & 3 & 4 & 5 & 5 \end{bmatrix}
\]
\[
J = \begin{bmatrix}1 & 2 & 1 & 2 & 3 & 5 & 4 & 3 & 5 \end{bmatrix}
\]
\[
V = \begin{bmatrix}7 & 1.1 & 1.1 & 2 & 3 & 3 & 0.5 & 3 & 4.2 \end{bmatrix}
\]

\subsection{Opis rešitve}
Implementacija ponuja tri metode: $\textit{nep\_chol}$, ki izračuna nepopolni razcep Choleskega za podano matriko $A$, $\textit{conj\_grad\_baseline}$, ki izvede metodo konjugiranih gradientov nad podano matriko $A$ in vektorjem desnih strani $b$ in vrne iskan $x$, in $\textit{conj\_grad}$,  ki prav tako vrne $x$, le da metodo konjugiranih gradientov izvede s predpogojevanjem s podano  spodnjetrikotno matriko $L$.


\subsubsection{Nepopolni razcep Choleskega}

\subsubsection{Metoda konjugiranih gradientov brez predpogojevanja}
Iščemo rešitev sistema $Ax=b$, kjer je matrika $A$ simetrična in pozitivno definitna. Za začetni približek $x_0$ vzamemo kar ničelni vektor in izračunamo $r_0 = b - Ax_0$. Nastavimo še $p_0=x_0$ in nato ponavljamo sledečo iteracijo:


\[
\alpha_k=\frac{r_k^Tr_k}{p_k^TAp_k}
\]

\[
x_{k+1}=x_k-\alpha_kp_k
\]

\[
r_{k+1}=r_k-\alpha_kAp_k
\]

\[
\beta_k=\frac{r_{k+1}^Tr_{k+1}}{r_k^Tr_k}
\]

\[
p_{k+1}=r_{k+1}+\beta_kp_k
\]

Postopek zaključimo kadar je norma vektorja $r_{k+1}$ manjša od želene tolerance.


\subsubsection{Metoda konjugiranih gradientov s predpogojevanjem}
Predpogojevanje izvajamo z matriko $M=LL^T$, kjer spodnjetrikotno matriko $L$ dobimo z nepopolnim razcepom Choleskega. Rešiti moramo sistem $Mz=r$, vendar ker nočemo računati inverza $M^{-1}$ (iz vidika časovne zahtevnosti), lahko najprej s premo substitucijo rešimo sistem $La=r$ in nato z obratno substitucijo rešimo sistem $L^Tz=a$.  Tako prema kot obratna substitucija imata časovno zahtevnost $O(n^2)$, kar je bistveno hitrejše od računanja inverza.


Zgornji postopek moramo ponoviti v vsakem koraku iteracije, in sicer na sledeč način: najprej kot začetno vrednost $p_0$ nastavimo rešitev sistema $Mz_0=r_0$, korake pa izvajamo na sledeč način:


\[
\alpha_k=\frac{r_k^Tz_k}{p_k^TAp_k}
\]

\[
x_{k+1}=x_k-\alpha_kp_k
\]

\[
r_{k+1}=r_k-\alpha_kAp_k
\]

Rešimo sistem $Mz_{k+1}=r_{k+1}$.


\[
\beta_k=\frac{r_{k+1}^Tz_{k+1}}{r_k^Tz_k}
\]

\[
p_{k+1}=z_{k+1}+\beta_kp_k
\]

Zaustavitveni pogoj ostane enak kot pri osnovni metodi.


\subsection{Primer uporabe}
Spodnji izsek kode demonstrira razliko med metodo konjugiranih gradientov brez predpogojevanja in s predpogojevanjem. 


\begin{lstlisting}
(*@\HLJLk{using}@*) (*@\HLJLn{MKG}@*)(*@\HLJLp{,}@*) (*@\HLJLn{Plots}@*)(*@\HLJLp{,}@*) (*@\HLJLn{LinearAlgebra}@*)(*@\HLJLp{,}@*) (*@\HLJLn{SparseArrays}@*)

(*@\HLJLn{I}@*) (*@\HLJLoB{=}@*) (*@\HLJLp{[}@*)(*@\HLJLnfB{1.}@*)(*@\HLJLp{,}@*) (*@\HLJLni{1}@*)(*@\HLJLp{,}@*) (*@\HLJLni{2}@*)(*@\HLJLp{,}@*) (*@\HLJLni{2}@*)(*@\HLJLp{,}@*) (*@\HLJLni{3}@*)(*@\HLJLp{,}@*) (*@\HLJLni{3}@*)(*@\HLJLp{,}@*) (*@\HLJLni{4}@*)(*@\HLJLp{,}@*) (*@\HLJLni{5}@*)(*@\HLJLp{,}@*) (*@\HLJLni{5}@*)(*@\HLJLp{]}@*)
(*@\HLJLn{J}@*) (*@\HLJLoB{=}@*) (*@\HLJLp{[}@*)(*@\HLJLnfB{1.}@*)(*@\HLJLp{,}@*) (*@\HLJLni{2}@*)(*@\HLJLp{,}@*) (*@\HLJLni{1}@*)(*@\HLJLp{,}@*) (*@\HLJLni{2}@*)(*@\HLJLp{,}@*) (*@\HLJLni{3}@*)(*@\HLJLp{,}@*) (*@\HLJLni{5}@*)(*@\HLJLp{,}@*) (*@\HLJLni{4}@*)(*@\HLJLp{,}@*) (*@\HLJLni{3}@*)(*@\HLJLp{,}@*) (*@\HLJLni{5}@*)(*@\HLJLp{]}@*)
(*@\HLJLn{V}@*) (*@\HLJLoB{=}@*) (*@\HLJLp{[}@*)(*@\HLJLnfB{7.}@*)(*@\HLJLp{,}@*) (*@\HLJLnfB{1.1}@*)(*@\HLJLp{,}@*) (*@\HLJLnfB{1.1}@*)(*@\HLJLp{,}@*) (*@\HLJLni{2}@*)(*@\HLJLp{,}@*) (*@\HLJLni{3}@*)(*@\HLJLp{,}@*) (*@\HLJLni{3}@*)(*@\HLJLp{,}@*) (*@\HLJLnfB{0.5}@*)(*@\HLJLp{,}@*) (*@\HLJLni{3}@*)(*@\HLJLp{,}@*) (*@\HLJLnfB{4.2}@*)(*@\HLJLp{]}@*)
(*@\HLJLn{A}@*) (*@\HLJLoB{=}@*) (*@\HLJLnf{sparse}@*)(*@\HLJLp{(}@*)(*@\HLJLn{I}@*)(*@\HLJLp{,}@*) (*@\HLJLn{J}@*)(*@\HLJLp{,}@*) (*@\HLJLn{V}@*)(*@\HLJLp{)}@*)
(*@\HLJLn{b}@*) (*@\HLJLoB{=}@*) (*@\HLJLp{[}@*)(*@\HLJLni{2}@*)(*@\HLJLp{,}@*) (*@\HLJLni{3}@*)(*@\HLJLp{,}@*) (*@\HLJLoB{-}@*)(*@\HLJLni{5}@*)(*@\HLJLp{,}@*) (*@\HLJLni{1}@*)(*@\HLJLp{,}@*) (*@\HLJLnfB{0.2}@*)(*@\HLJLp{]}@*)
(*@\HLJLn{L}@*) (*@\HLJLoB{=}@*) (*@\HLJLnf{nep{\_}chol}@*)(*@\HLJLp{(}@*)(*@\HLJLn{A}@*)(*@\HLJLp{)}@*)
(*@\HLJLn{x1}@*)(*@\HLJLp{,}@*) (*@\HLJLn{it1}@*)(*@\HLJLp{,}@*) (*@\HLJLn{res1}@*) (*@\HLJLoB{=}@*) (*@\HLJLnf{conj{\_}grad{\_}baseline}@*)(*@\HLJLp{(}@*)(*@\HLJLn{A}@*)(*@\HLJLp{,}@*) (*@\HLJLn{b}@*)(*@\HLJLp{,}@*) (*@\HLJLn{vrniresid}@*)(*@\HLJLoB{=}@*)(*@\HLJLkc{true}@*)(*@\HLJLp{)}@*)
(*@\HLJLn{x2}@*)(*@\HLJLp{,}@*) (*@\HLJLn{it2}@*)(*@\HLJLp{,}@*) (*@\HLJLn{res2}@*) (*@\HLJLoB{=}@*) (*@\HLJLnf{conj{\_}grad}@*)(*@\HLJLp{(}@*)(*@\HLJLn{A}@*)(*@\HLJLp{,}@*) (*@\HLJLn{b}@*)(*@\HLJLp{,}@*) (*@\HLJLn{L}@*)(*@\HLJLp{,}@*) (*@\HLJLn{vrniresid}@*)(*@\HLJLoB{=}@*)(*@\HLJLkc{true}@*)(*@\HLJLp{)}@*)

(*@\HLJLnf{println}@*)(*@\HLJLp{(}@*)(*@\HLJLs{"{}MKG}@*) (*@\HLJLs{brez}@*) (*@\HLJLs{predpogojevanja}@*) (*@\HLJLs{se}@*) (*@\HLJLs{zaustavi}@*) (*@\HLJLs{po}@*) (*@\HLJLsi{{\$}it1}@*) (*@\HLJLs{korakih."{}}@*)(*@\HLJLp{)}@*)
(*@\HLJLnf{println}@*)(*@\HLJLp{(}@*)(*@\HLJLs{"{}MKG}@*) (*@\HLJLs{s}@*) (*@\HLJLs{predpogojevanjem}@*) (*@\HLJLs{se}@*) (*@\HLJLs{zaustavi}@*) (*@\HLJLs{po}@*) (*@\HLJLsi{{\$}it2}@*) (*@\HLJLs{korakih."{}}@*)(*@\HLJLp{)}@*)
\end{lstlisting}

\begin{lstlisting}
MKG brez predpogojevanja se zaustavi po 5 korakih.
MKG s predpogojevanjem se zaustavi po 1 korakih.
\end{lstlisting}


Kot vidimo že na relativno majhnem primeru, predpogojevanje bistveno izboljša konvergenco.


V spodnjem izseku si ogledamo še primer z bistveno večjo matriko, ki jo zgeneriramo psevdonaključno: za vse diagonalne elemente zgeneriramo naključna števila iz nekega intervala, za preostale elemente pa  z neko manjšo verjetnostjo (npr. 0.1) zgeneriramo manjša števila, da ohranimo dominantnost diagonale. Vsako izmed teh števil zapišemo tako na indeks $(i,j)$ kot $(j,i)$, da ohranimo simetričnost. Ta postopek sicer ne zagotavlja pozitivne definitnosti zgenerirane matrike, vendar se empirično izkaže kot dovolj dober za potrebe te demonstracije.


Z zastavico $\textit{vrniresid}$ nam metodi vrneta tabelo residualov (oziroma neskončnih norm vektorjev $r_{k+1}$) na posameznem koraku iteracije. Residuale izrišemo s paketom Plots in tako dobimo vizualno primerjavo hitrosti konvergence med metodama. 


\begin{lstlisting}
(*@\HLJLn{n}@*)(*@\HLJLoB{=}@*)(*@\HLJLni{700}@*)
(*@\HLJLn{I}@*) (*@\HLJLoB{=}@*) (*@\HLJLp{[}@*)(*@\HLJLnfB{1.0}@*)(*@\HLJLp{]}@*)
(*@\HLJLn{J}@*) (*@\HLJLoB{=}@*) (*@\HLJLp{[}@*)(*@\HLJLnfB{1.0}@*)(*@\HLJLp{]}@*)
(*@\HLJLn{V}@*) (*@\HLJLoB{=}@*) (*@\HLJLp{[}@*)(*@\HLJLnf{rand}@*)(*@\HLJLp{()}@*)(*@\HLJLoB{*}@*)(*@\HLJLnfB{10.0}@*)(*@\HLJLp{]}@*)
(*@\HLJLk{for}@*) (*@\HLJLn{i}@*)(*@\HLJLoB{=}@*)(*@\HLJLni{2}@*)(*@\HLJLoB{:}@*)(*@\HLJLn{n}@*)
    (*@\HLJLnf{push!}@*)(*@\HLJLp{(}@*)(*@\HLJLn{V}@*)(*@\HLJLp{,}@*) (*@\HLJLnf{rand}@*)(*@\HLJLp{()}@*)(*@\HLJLoB{*}@*)(*@\HLJLnfB{80.0}@*)(*@\HLJLp{)}@*)
    (*@\HLJLnf{push!}@*)(*@\HLJLp{(}@*)(*@\HLJLn{I}@*)(*@\HLJLp{,}@*) (*@\HLJLn{i}@*)(*@\HLJLp{)}@*)
    (*@\HLJLnf{push!}@*)(*@\HLJLp{(}@*)(*@\HLJLn{J}@*)(*@\HLJLp{,}@*) (*@\HLJLn{i}@*)(*@\HLJLp{)}@*)

    (*@\HLJLk{if}@*) (*@\HLJLnf{rand}@*)(*@\HLJLp{()}@*) (*@\HLJLoB{<}@*) (*@\HLJLnfB{0.1}@*)
        (*@\HLJLn{r}@*) (*@\HLJLoB{=}@*) (*@\HLJLnf{rand}@*)(*@\HLJLp{()}@*)
        (*@\HLJLnf{push!}@*)(*@\HLJLp{(}@*)(*@\HLJLn{V}@*)(*@\HLJLp{,}@*) (*@\HLJLn{r}@*)(*@\HLJLp{)}@*)
        (*@\HLJLnf{push!}@*)(*@\HLJLp{(}@*)(*@\HLJLn{V}@*)(*@\HLJLp{,}@*) (*@\HLJLn{r}@*)(*@\HLJLp{)}@*)
        (*@\HLJLn{r1}@*) (*@\HLJLoB{=}@*) (*@\HLJLnf{rand}@*)(*@\HLJLp{(}@*)(*@\HLJLni{1}@*)(*@\HLJLoB{:}@*)(*@\HLJLn{n}@*)(*@\HLJLp{)}@*)
        (*@\HLJLn{r2}@*) (*@\HLJLoB{=}@*) (*@\HLJLnf{rand}@*)(*@\HLJLp{(}@*)(*@\HLJLni{1}@*)(*@\HLJLoB{:}@*)(*@\HLJLn{n}@*)(*@\HLJLp{)}@*)
        (*@\HLJLk{while}@*) (*@\HLJLn{r1}@*)(*@\HLJLoB{==}@*)(*@\HLJLn{i}@*) (*@\HLJLoB{||}@*) (*@\HLJLn{r2}@*)(*@\HLJLoB{==}@*)(*@\HLJLn{i}@*) (*@\HLJLcs{{\#}}@*) (*@\HLJLcs{poskrbimo,}@*) (*@\HLJLcs{da}@*) (*@\HLJLcs{ne}@*) (*@\HLJLcs{prepišemo}@*) (*@\HLJLcs{diagonale}@*)
            (*@\HLJLn{r1}@*) (*@\HLJLoB{=}@*) (*@\HLJLnf{rand}@*)(*@\HLJLp{(}@*)(*@\HLJLni{1}@*)(*@\HLJLoB{:}@*)(*@\HLJLn{n}@*)(*@\HLJLp{)}@*)
            (*@\HLJLn{r2}@*) (*@\HLJLoB{=}@*) (*@\HLJLnf{rand}@*)(*@\HLJLp{(}@*)(*@\HLJLni{1}@*)(*@\HLJLoB{:}@*)(*@\HLJLn{n}@*)(*@\HLJLp{)}@*)
        (*@\HLJLk{end}@*)
        (*@\HLJLnf{push!}@*)(*@\HLJLp{(}@*)(*@\HLJLn{I}@*)(*@\HLJLp{,}@*) (*@\HLJLn{r1}@*)(*@\HLJLp{)}@*)
        (*@\HLJLnf{push!}@*)(*@\HLJLp{(}@*)(*@\HLJLn{J}@*)(*@\HLJLp{,}@*) (*@\HLJLn{r2}@*)(*@\HLJLp{)}@*)
        (*@\HLJLnf{push!}@*)(*@\HLJLp{(}@*)(*@\HLJLn{I}@*)(*@\HLJLp{,}@*) (*@\HLJLn{r2}@*)(*@\HLJLp{)}@*)
        (*@\HLJLnf{push!}@*)(*@\HLJLp{(}@*)(*@\HLJLn{J}@*)(*@\HLJLp{,}@*) (*@\HLJLn{r1}@*)(*@\HLJLp{)}@*)
    (*@\HLJLk{end}@*)
(*@\HLJLk{end}@*)
(*@\HLJLn{A}@*) (*@\HLJLoB{=}@*) (*@\HLJLnf{sparse}@*)(*@\HLJLp{(}@*)(*@\HLJLn{I}@*)(*@\HLJLp{,}@*) (*@\HLJLn{J}@*)(*@\HLJLp{,}@*) (*@\HLJLn{V}@*)(*@\HLJLp{)}@*)
(*@\HLJLn{b}@*) (*@\HLJLoB{=}@*) (*@\HLJLnf{rand}@*)(*@\HLJLp{(}@*)(*@\HLJLn{n}@*)(*@\HLJLp{)}@*)
(*@\HLJLn{x1}@*)(*@\HLJLp{,}@*) (*@\HLJLn{it1}@*)(*@\HLJLp{,}@*) (*@\HLJLn{res1}@*) (*@\HLJLoB{=}@*) (*@\HLJLnf{conj{\_}grad{\_}baseline}@*)(*@\HLJLp{(}@*)(*@\HLJLn{A}@*)(*@\HLJLp{,}@*) (*@\HLJLn{b}@*)(*@\HLJLp{,}@*) (*@\HLJLn{vrniresid}@*)(*@\HLJLoB{=}@*)(*@\HLJLkc{true}@*)(*@\HLJLp{,}@*) (*@\HLJLn{tol}@*)(*@\HLJLoB{=}@*)(*@\HLJLnfB{10e-20}@*)(*@\HLJLp{)}@*)
(*@\HLJLn{L}@*) (*@\HLJLoB{=}@*) (*@\HLJLnf{nep{\_}chol}@*)(*@\HLJLp{(}@*)(*@\HLJLn{A}@*)(*@\HLJLp{)}@*)
(*@\HLJLn{x2}@*)(*@\HLJLp{,}@*) (*@\HLJLn{it2}@*)(*@\HLJLp{,}@*) (*@\HLJLn{res2}@*) (*@\HLJLoB{=}@*) (*@\HLJLnf{conj{\_}grad}@*)(*@\HLJLp{(}@*)(*@\HLJLn{A}@*)(*@\HLJLp{,}@*) (*@\HLJLn{b}@*)(*@\HLJLp{,}@*) (*@\HLJLn{L}@*)(*@\HLJLp{,}@*) (*@\HLJLn{vrniresid}@*)(*@\HLJLoB{=}@*)(*@\HLJLkc{true}@*)(*@\HLJLp{,}@*) (*@\HLJLn{tol}@*)(*@\HLJLoB{=}@*)(*@\HLJLnfB{10e-20}@*)(*@\HLJLp{)}@*)

(*@\HLJLnf{plot}@*)(*@\HLJLp{(}@*)(*@\HLJLn{res1}@*)(*@\HLJLp{,}@*) (*@\HLJLn{label}@*)(*@\HLJLoB{=}@*)(*@\HLJLs{"{}brez}@*) (*@\HLJLs{predpogojevanja"{}}@*)(*@\HLJLp{,}@*) (*@\HLJLn{title}@*)(*@\HLJLoB{=}@*)(*@\HLJLs{"{}Primerjava}@*) (*@\HLJLs{residualov"{}}@*)(*@\HLJLp{)}@*)
(*@\HLJLnf{plot!}@*)(*@\HLJLp{(}@*)(*@\HLJLn{res2}@*)(*@\HLJLp{,}@*) (*@\HLJLn{label}@*)(*@\HLJLoB{=}@*)(*@\HLJLs{"{}s}@*) (*@\HLJLs{predpogojevanjem"{}}@*)(*@\HLJLp{,}@*) (*@\HLJLn{title}@*)(*@\HLJLoB{=}@*)(*@\HLJLs{"{}Primerjava}@*) (*@\HLJLs{residualov"{}}@*)(*@\HLJLp{)}@*)
\end{lstlisting}

\includegraphics[width=\linewidth]{jl_2WfWuh/demo_1_1.pdf}


\end{document}
